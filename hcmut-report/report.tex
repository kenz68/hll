% 'draft' mode can be used to speed up compilation
\documentclass[twoside,final]{hcmut-report}
\usepackage{codespace}

% Draft watermark
% https://github.com/callegar/LaTeX-draftwatermark

% Encodings
\usepackage{gensymb,textcomp}

% Better tables
% Wide tables go to https://tex.stackexchange.com/q/332902
\usepackage{array,longtable,multicol,multirow,siunitx,tabularx}

% Better enum
\usepackage{enumitem}

% Graphics
\usepackage{caption,float}

% Add options for figures, like max width, framing, etc.
\usepackage[export]{adjustbox}

% References
% Use \cref{} or \Cref{} instead of \ref{}
\usepackage[nameinlink]{cleveref}

% FOR DEMONSTRATION PURPOSES, REMOVE IN PRODUCTION
\usepackage{mwe}

% Sub-preambles
% https://github.com/MartinScharrer/standalone

% Configurations
\coursename{Course name h}
\reporttype{Report type h}
\title{Report title h}
\advisor{& Advisor h &}
\stuname{%
  & Student 1 & ID 1 \\
  & Student 2 & ID 2 \\
  & Student 3 & ID 3 \\
}

% Allow page breaks inside align* environment
%\allowdisplaybreaks{}

% Makes a lot of things blue, avoid at all costs
%\everymath{\color{blue}}

% Set depth of numbering for counters
\AtBeginDocument{\counterwithin{lstlisting}{section}}

% Rename some sections
%\AtBeginDocument{\renewcommand*{\contentsname}{Contents}}
%\AtBeginDocument{\renewcommand*{\refname}{References}}
%\AtBeginDocument{\renewcommand*{\bibname}{References}}

% Custom commands
%\newcommand*\mean[1]{\bar{#1}}

\begin{document}
\coverpage%

%\section*{Member list \& Workload}
%\newcounter{memberrowno}
%\setcounter{memberrowno}{0}
%\begin{center}
%  \begin{tabular}{>{\stepcounter{memberrowno}\thememberrowno}llcc}
%    \toprule
%    \multicolumn{1}{c}{\textbf{No.}} & \textbf{Full name} & \textbf{Student ID} & \textbf{Contribution} \\
%    \midrule
%                                     & h                  & xxxxxxx             & 100\%                       \\
%                                     & h                  & xxxxxxx             & 100\%                       \\
%    \bottomrule
%  \end{tabular}
%\end{center}
%\clearpage

\tableofcontents
\listoffigures
\listoftables
\lstlistoflistings{}

\clearpage
\section{Normal section}
This is how you normally work with \LaTeX, but you can also split a project into smaller files for easier management.
To import other files, you can use \mintinline{latex}{\input{}} or \mintinline{latex}{\include{}}.
There differences can be found at \url{https://tex.stackexchange.com/a/250}, but in short

\begin{center}
  \mintinline{latex}{\include{filename}} = \mintinline{latex}{\clearpage \input{filename} \clearpage}
\end{center}

\section{Label prefixes}
There are no definite rules for label prefixes, but you can use the following as a guideline.
\begin{itemize}
  \item \textbf{chap:} for chapters
  \item \textbf{sec:} for sections
  \item \textbf{subsec:} for subsections
  \item \textbf{eq:} for equations
  \item \textbf{fig:} for figures
  \item \textbf{tab:} for tables
  \item \textbf{enum:} for enumerators and items
  \item \textbf{fn:} for footnotes
  \item \textbf{lst:} for listings
  \item \textbf{alg:} for algorithms
  \item \textbf{app:} for appendices
\end{itemize}

The \mintinline{latex}{\caption} macro increases the used counter and sets the current label text which is used by \mintinline{latex}{\label}.
If you use \mintinline{latex}{\label} before it the old label text is used instead, which leads to a wrong number.
Always use \mintinline{latex}{\label} after \mintinline{latex}{\caption} and not before or in it.

That said, conventions are just conventions, and you can use whatever you want as long as you are consistent.

\clearpage
\section{Better tables}
The recommended way is by using the \mintinline{latex}{booktabs} package and drop all vertical rules.

\mintinline{latex}{tabularx} is simply tabular but with X environment, meaning that it will try to use all of \mintinline{latex}{\linewidth}.

\begin{table}[H]
  \centering
  \caption{Tabularx table}%
  \label{tab:tabularx}
  \begin{tabularx}{\linewidth}{l*{2}{X}}
    \toprule
         & OOP & FP \\
    \cmidrule(lr){2-3}
    Pros &     &    \\
         &     &    \\
         &     &    \\
    \midrule
    Cons &     &    \\
         &     &    \\
         &     &    \\
    \bottomrule
  \end{tabularx}
\end{table}

More information can be found at \url{https://latex-tutorial.com/tables-in-latex/}.

In \mintinline{latex}{tabular}, with the \mintinline{latex}{p}, \mintinline{latex}{m} or \mintinline{latex}{b} column types, sometimes you will notice that the width of the table is wider than the sum of the widths of the columns.
This is due to the padding added by the \mintinline{latex}{\tabcolsep} and the line width of the vertical separators which are added by default.

\begin{code}{latex}
  \tabcolsep + p{length} + \tabcolsep
\end{code}

By default, \mintinline{latex}{tabcolsep} is set to 6pt, which equals to 2.12mm in digital printing.
The use of \mintinline{latex}{@{}..@{}} voids this behavior.

Additionally, if you have to insert a very long table, which takes up two or more pages in your document, use the \mintinline{latex}{longtable} package.

\begin{longtable}[H]{|m{0.3\linewidth}|m{0.3\linewidth}|}
  \caption{Long table caption}\label{tab:long}          \\
  \toprule
  \multicolumn{2}{|c|}{Begin of Table}                  \\
  \midrule
  Something     & something else                        \\
  \midrule
  \endfirsthead

  \toprule
  \multicolumn{2}{|c|}{Continuation of \Cref{tab:long}} \\
  \midrule
  Something     & something else                        \\
  \midrule
  \endhead

  \midrule
  \endfoot

  \midrule
  \multicolumn{2}{|c|}{End of Table}                    \\
  \bottomrule
  \endlastfoot

  Lots of lines & like this                             \\
  Lots of lines & like this                             \\
  Lots of lines & like this                             \\
  Lots of lines & like this                             \\
  Lots of lines & like this                             \\
  Lots of lines & like this                             \\
  Lots of lines & like this                             \\
  Lots of lines & like this                             \\
  Lots of lines & like this                             \\
  Lots of lines & like this                             \\
  Lots of lines & like this                             \\
  Lots of lines & like this                             \\
  Lots of lines & like this                             \\
  Lots of lines & like this                             \\
  Lots of lines & like this                             \\
  Lots of lines & like this                             \\
  Lots of lines & like this                             \\
  Lots of lines & like this                             \\
  Lots of lines & like this                             \\
  Lots of lines & like this                             \\
  Lots of lines & like this                             \\
  Lots of lines & like this                             \\
  Lots of lines & like this                             \\
  Lots of lines & like this                             \\
  Lots of lines & like this                             \\
  Lots of lines & like this                             \\
  Lots of lines & like this                             \\
  Lots of lines & like this                             \\
  Lots of lines & like this                             \\
  Lots of lines & like this                             \\
  Lots of lines & like this                             \\
  Lots of lines & like this                             \\
  Lots of lines & like this                             \\
  Lots of lines & like this                             \\
  Lots of lines & like this                             \\
  Lots of lines & like this                             \\
  Lots of lines & like this                             \\
  Lots of lines & like this                             \\
  Lots of lines & like this                             \\
  Lots of lines & like this                             \\
  Lots of lines & like this                             \\
  Lots of lines & like this                             \\
  Lots of lines & like this                             \\
  Lots of lines & like this                             \\
  Lots of lines & like this                             \\
  Lots of lines & like this                             \\
  Lots of lines & like this                             \\
  Lots of lines & like this                             \\
  Lots of lines & like this                             \\
  Lots of lines & like this                             \\
  Lots of lines & like this                             \\
  Lots of lines & like this                             \\
  Lots of lines & like this                             \\
  Lots of lines & like this                             \\
  Lots of lines & like this                             \\
  Lots of lines & like this                             \\
  Lots of lines & like this                             \\
  Lots of lines & like this                             \\
  Lots of lines & like this                             \\
  Lots of lines & like this                             \\
  Lots of lines & like this                             \\
  Lots of lines & like this                             \\
  Lots of lines & like this                             \\
\end{longtable}

You may use the \mintinline{latex}{\label} command so that you can cross reference longtables with \mintinline{latex}{\ref}.
Note however, that the \mintinline{latex}{\label} command should not be used in a heading that may appear more than once.
Place it either in the firsthead, or in the body of the table.
It should not be the first command in any entry.

\section{Better enumerator}
Normal enumerator gets the job done, but what if you want custom numbering?
This implementation allows custom labeling, either by pre-defined rules or in-place.

\begin{enumerate}[start=4,label={\alph*.yeah}]
  \item First item
  \item Second item
  \item[custom] Third item
\end{enumerate}

\section{Codespace}
\subsection{Listings}
This is the recommended way to insert simple code.

\begin{itemize}
  \item \lstinputlisting[
    language=python,
    caption={External import},name=ext-import,label=lst:ext-import
  ]{code/example.py}

  \item \lstinputlisting[
    firstline=10,lastline=13,language=python,
    caption={External import but with a line range},
    name=line-range,label=lst:line-range
  ]{code/example.py}

  \item \begin{lstlisting}[
    language=python,caption={Embedded},name=embedded,label=lst:embedded
  ]
  from typing import Iterator

  # This is an example
  class Math:
      @staticmethod
      def fib(n: int) -> Iterator[int]:
          """Fibonacci series up to n."""
          a, b = 0, 1
          while a < n:
              yield a
              a, b = b, a + b

  result = sum(Math.fib(42))
  print("The answer is {}".format(result))
  \end{lstlisting}

  \item Inline

  \lstinline[
    language=python,
    name=inline,label=lst:inline
  ]{print('Hello, world!')}
\end{itemize}

\subsection{Minted}
This provide better looking code, but requires external setup:

\emph{Minted requires python Pygments and the \mintinline{text}{--shell-escape} flag.}

\begin{itemize}
  \item External import

  \inputcode[highlightlines={1,10-13}]{Python}{code/example.py}

  \item With a line range

  \inputcode[firstnumber=1,firstline=10,lastline=13]{Python}{code/example.py}

  \item Embedded

  \begin{code}{python}
  from typing import Iterator

  # This is an example
  class Math:
      @staticmethod
      def fib(n: int) -> Iterator[int]:
          """Fibonacci series up to n."""
          a, b = 0, 1
          while a < n:
              yield a
              a, b = b, a + b

  result = sum(Math.fib(42))
  print("The answer is {}".format(result))
  \end{code}

  \item Inline

  \mintinline{Python}{print('Hello, world!')}
\end{itemize}

\section{Figures with flexible width}

\hrule % to see \linewidth

\begin{figure}[H]
  \includegraphics[max width=0.9\linewidth]{example-image-1x1}
  \caption{Example image 1x1}%
  \label{fig:example-image-1x1}
\end{figure}

\begin{figure}[H]
  \includegraphics[frame,scale=0.7]{example-image-a4}
  \caption{Example image A4}%
  \label{fig:example-image-a4}
\end{figure}

With \mintinline{latex}{\adjincludegraphics} (or \mintinline{latex}{\adjustimage}) you can also use the original width as \mintinline{latex}{\width}:

\begin{figure}[H]
  \adjincludegraphics[width=\ifdim \width > \linewidth \linewidth \else \width \fi]{example-image}
  \caption{Set figure width imperatively}%
  \label{fig:imperative-width}
\end{figure}


\bibliographystyle{plain}
\bibliography{refs/example.bib}
\nocite{*}

\end{document}
